\section{Deeper Understanding of the Data}

Before filtering, the dataset had roughly 94\% failed undercuts and 6\% successful ones due to cases with unrealistic gaps. After applying the 2-second filter, the overall success rate increased to roughly 10.1\%, producing a more realistic strategic dataset. This filtering step improves dataset quality by aligning with what physics and race strategy allow.

Most successful undercuts occur when the attacking driver is 0.5--1.5 seconds behind before pitting, while the majority of attempts cluster between 0.74 and 1.54 seconds after filtering (middle 50\% of attempts). Tire age shows the middle 50\% of attempts occur at 6--17 laps on tires, and tire age differentials often influence strategy more than absolute ages. Pace differential varies in a wide range from $-2$ to $+2$ seconds per lap, and many undercuts are attempted even when the attacker is slower, showing that teams often pit preemptively to avoid traffic.

Correlations with undercut success are weak: gap ($-0.03$), attacker tire age ($-0.01$), pace differential ($-0.004$), pit duration ($+0.14$). This suggests nonlinear models may later outperform linear ones.

From 2014 to 2024, undercut success rates vary between 5.2\% and 13.6\% without strong trends. Circuit effects are much more pronounced, ranging from about 25\% at Monaco to about 34.5\% at Circuit Gilles Villeneuve, indicating that pit lane length, tire degradation and overtaking difficulty play major roles.

\section{Visualizations}

The notebook includes the following visualizations: class distribution charts, histograms for gaps and tire age, a correlation heatmap, year-over-year success rate plots, circuit-specific success charts, ROC curves, precision-recall curves, and confusion matrices. Each visual is labeled and includes explanatory context.

\section{Noteworthy Findings}

The 2-second gap filter is crucial for obtaining realistic data and removes many non-strategic pit stops. Gap is the most important individual feature but is insufficient alone; tire age differential and circuit characteristics matter significantly. Circuits exhibit large variability, with approximately three-fold differences in success rates between top and bottom circuits. Pit stop duration shows positive correlation ($+0.14$), indicating faster pit stops for the attacker increase success probability. Roughly one-third of attempts occur when the attacker is slower, reflecting preemptive strategy.

